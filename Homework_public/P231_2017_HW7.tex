\documentclass[12pt]{article}
%% arXiv paper template by Flip Tanedo
%% last updated: Dec 2016



%%%%%%%%%%%%%%%%%%%%%%%%%%%%%
%%%  THE USUAL PACKAGES  %%%%
%%%%%%%%%%%%%%%%%%%%%%%%%%%%%

\usepackage{amsmath}
\usepackage{amssymb}
\usepackage{amsfonts}
\usepackage{graphicx}
\usepackage{xcolor}
\usepackage{nopageno}
\usepackage{enumerate}
\usepackage{parskip}

%%%%%%%%%%%%%%%%%%%%%%%%%%%%%%%%%
%%%  UNUSUAL PACKAGES        %%%%
%%%  Uncomment as necessary. %%%%
%%%%%%%%%%%%%%%%%%%%%%%%%%%%%%%%%

\usepackage{tikzfeynman}

\usepackage{titlesec}
\titleformat*{\section}{\large\bfseries}

%% MATH AND PHYSICS SYMBOLS
%% ------------------------
%\usepackage{slashed}       % \slashed{k}
%\usepackage{mathrsfs}      % Weinberg-esque letters
%\usepackage{youngtab}	    % Young Tableaux
%\usepackage{pifont}        % check marks
\usepackage{bbm}           % \mathbbm{1} incomp. w/ XeLaTeX 
%\usepackage[normalem]{ulem} % for \sout


%% CONTENT FORMAT AND DESIGN (below for general formatting)
%% --------------------------------------------------------
\usepackage{lipsum}        % block of text (formatting test)
%\usepackage{color}         % \color{...}, colored text
%\usepackage{framed}        % boxed remarks
%\usepackage{subcaption}    % subfigures; subfig depreciated
%\usepackage{paralist}      % compactitem
%\usepackage{appendix}      % subappendices
%\usepackage{cite}          % group cites (conflict: collref)
%\usepackage{tocloft}       % Table of Contents	

%% TABLES IN LaTeX
%% ---------------
%\usepackage{booktabs}      % professional tables
%\usepackage{nicefrac}      % fractions in tables,
%\usepackage{multirow}      % multirow elements in a table
%\usepackage{arydshln} 	    % dashed lines in arrays

%% Other Packages and Notes
%% ------------------------
%\usepackage[font=small]{caption} % caption font is small



%\renewcommand{\thesection}{}
%\renewcommand{\thesubsection}{\arabic{subsection}}

%%%%%%%%%%%%%%%%%%%%%%%%%%%%%%%%%%%%%%%%%%%%%%%
%%%  PAGE FORMATTING and (RE)NEW COMMANDS  %%%%
%%%%%%%%%%%%%%%%%%%%%%%%%%%%%%%%%%%%%%%%%%%%%%%

\usepackage[margin=2cm]{geometry}   % reasonable margins

\graphicspath{{figures/}}	        % set directory for figures

% for capitalized things
\newcommand{\acro}[1]{\textsc{\MakeLowercase{#1}}}    

\numberwithin{equation}{section}    % set equation numbering
\renewcommand{\tilde}{\widetilde}   % tilde over characters
\renewcommand{\vec}[1]{\mathbf{#1}} % vectors are boldface

\newcommand{\dbar}{d\mkern-6mu\mathchar'26}    % for d/2pi
\newcommand{\ket}[1]{\left|#1\right\rangle}    % <#1|
\newcommand{\bra}[1]{\left\langle#1\right|}    % |#1>
\newcommand{\Xmark}{\text{\sffamily X}}        % cross out

% Change list spacing (instead of package paralist)
% from: http://en.wikibooks.org/wiki/LaTeX/List_Structures#Line_spacing
%\let\oldenumerate\enumerate
%\renewcommand{\enumerate}{
%  \oldenumerate
%  \setlength{\itemsep}{1pt}
%  \setlength{\parskip}{0pt}
%  \setlength{\parsep}{0pt}
%}

\let\olditemize\itemize
\renewcommand{\itemize}{
  \olditemize
  \setlength{\itemsep}{1pt}
  \setlength{\parskip}{0pt}
  \setlength{\parsep}{0pt}
}


% Commands for temporary comments
\newcommand{\comment}[2]{\textcolor{red}{[\textbf{#1} #2]}}
\newcommand{\flip}[1]{{\color{red} [\textbf{Flip}: {#1}]}}
\newcommand{\email}[1]{\texttt{\href{mailto:#1}{#1}}}

\newenvironment{institutions}[1][2em]{\begin{list}{}{\setlength\leftmargin{#1}\setlength\rightmargin{#1}}\item[]}{\end{list}}


\usepackage{fancyhdr}		% to put preprint number



% Commands for listings package
\usepackage{listings}      % \begin{lstlisting}, for code
%
 \lstset{basicstyle=\ttfamily\footnotesize,breaklines=true}
%    sets style to small true-type


%%%%%%%%%%%%%%%%%%%%%%%%%%%%%%%%%%%%%%%%%%%%%%
%%%  TIKZ COMMANDS FOR EXTERNAL DIAGRAMS  %%%%
%%%  requires -shell-escape               %%%%
%%%  in texpad 1.7: prefs > shell esc sec %%%%
%%%%%%%%%%%%%%%%%%%%%%%%%%%%%%%%%%%%%%%%%%%%%%

%% This is for exporting tikz figures as into a ./tikz/ subfolder.
%% It is useful if you want pdf versions of the tikz diagrams or
%% if you need to speed up compilation of a large document with
%% many tikz diagrams.

%\write18{} % Careful with this!
%\usetikzlibrary{external}
%\tikzexternalize[prefix=tikz/] % folder for external pdfs


%%%%%%%%%%%%%%%%%%%
%%%  HYPERREF  %%%%
%%%%%%%%%%%%%%%%%%%

%% This package has to be at the end; can lead to conflicts
\usepackage{microtype}
\usepackage[
	colorlinks=true,
	citecolor=black,
	linkcolor=black,
	urlcolor=green!50!black,
	hypertexnames=false]{hyperref}



%%%%%%%%%%%%%%%%%%%%%
%%%  TITLE DATA  %%%%
%%%%%%%%%%%%%%%%%%%%%

%%% PREPRINT NUMBER USING fancyhdr
%%% Don't forget to set \thispagestyle{firststyle}
%%% ----------------------------------------------
%\renewcommand{\headrulewidth}{0pt} % no separator
%\fancypagestyle{firststyle}{
%\rhead{\footnotesize \texttt{UCI-TR-2016-XX}}}



\begin{document}

%\thispagestyle{empty}
%\thispagestyle{firststyle} %% to include preprint

\begin{center}

    {\Large \textsc{Homework 7:} 
    \textbf{Green's functions in diverse dimensions}}


    
\end{center}

\vskip .4cm

\noindent
\begin{tabular*}{\textwidth}{rlcrll}
	\textsc{Course:}& Physics 231, \emph{Methods of Theoretical Physics} (2017)
	&
%	\hspace{1.2cm}
	&
	\\
	\textsc{Instructor:}& Flip Tanedo (\email{flip.tanedo@ucr.edu})
	&
	%\hfill
	&
	& 
	\\
	\textsc{Due by:}& \emph{Monday}, November 27\footnote{Due date are flexible. There may be one more homework. All homework is due by the last lecture of our class Wednesday, Dec 6.}.
	&
	%\hfill
	&
	%	
\end{tabular*}



\textbf{Reading}: Enjoy Thanksgiving break\footnote{Things you may consider doing: if you have a car, you might consider going with a group of friends to Big Bear or Lake Arrowhead. If you prefer the ocean, you can check out the beaches in Orange County. If you go to Orange County, you might want to check out Sidecar Doughnuts in Costa Mesa---if you get there before 11am, they have the most amazing eggs benedict donut. If you prefer heading to Los Angeles, the Getty Museum has one of the best views of Los Angeles. In downtown you might enjoy Grand Central Market, which is not far from the Walt Disney Concert Hall and the new Broad museum. If you're staying in Riverside, there's a Turkey Trot 5K/15K run on Thanksgiving day.}. There will be a review lecture on Monday, November 20. \emph{No Lecture on Wednesday, November 22.} %
%
On November 27 we will leave Green's functions and will  focus on probability and statistics. A good reference is the book by Barlow. We will roughly follow the approach of Kyle Cranmer's excellent lectures\footnote{See \url{https://indico.cern.ch/event/243641/} and references therein.}. 


\emph{Why is this homework so long?!} You're being asked to re-do many of the steps that we've walked through in class. Refer back to the lecture notes as necessary.


\section{The 1D Damped Harmonic Oscillator: application}
% From Byron and Fuller, 7.3 

In Lecture 18 we solved for the Green's function of the damped harmonic oscillator:
\begin{align}
	\ddot x(t) + 2 \gamma \dot x(t) + \omega_0^2 x(t) = F(t) \ .
	\label{eq:ODE}
\end{align}
 We found:
\begin{align}
	G(t-t') &= \frac{
	e^{-\gamma(t-t')} 
	\sin\left[\sqrt{\omega_0^2-\gamma^2}\, (t-t')\right]
	}{\sqrt{\omega_0^2-\gamma^2}} \ .
\end{align}

Let's go back to the complete problem. We would like to find the solution to the forced, damped spring with some force function $F(t)$, that is, (\ref{eq:ODE}). The solutions of this system are $x(t)$, given by 
\begin{align}
	x(t) = A x_1(t) + B x_2(t) 
	+ 
	\int_{t_1}^{t_2}
	\frac{
	e^{-\gamma(t-t')}
	\sin \left[
	\sqrt{\omega_0^2 - \gamma^2} (t-t')
	\right]
	F(t')
	}{
	\sqrt{\omega_0^2 - \gamma^2}
	}
	dt' \ ,
	\label{eq:gen:sol}
\end{align}
where $x_{1,2}(t)$ are solutions to the \emph{homogeneous} differential equation, that is (\ref{eq:ODE}) with $F(t)=0$. 
%
Recall that we may take the upper limit of integration to be $t_2 = t$. The lower limit, $t_1$ can be taken to be the time at which the driving force is applied. 

\subsection{Initial Conditions}

The coefficients $A$ and $B$ are determined by the initial conditions of the problem. 
%
Even without knowing the precise form of $x_{1,2}(t)$, we know that for a specific initial condition (say, at $t_1$), the coefficients $A$ and $B$ vanish. What are the initial conditions for which this is true? Specify $x(t_1)$ and $\dot x(t_1)$. You should answer this by thinking about it physically, and confirm your intuition by looking at (\ref{eq:gen:sol}) to see that it makes sense. In a complete sentence (using words, not equations), explain the state of the system at $t=t_1$. 

\subsection{Exponentially falling blip}

For simplicity, set $t_1 = 0$. Suppose that the time-dependent force is
\begin{align}
	F(t) = F_0 e^{-\alpha t} \ .
\end{align}
Assume the initial conditions above for which $A=B=0$. Solve (\ref{eq:gen:sol}) by performing the integral. No need to do anything fancy like a contour integral. You may find it useful to write the sine as a sum of exponentials. Show that the solution is
\begin{align}
	x(t) = \frac{F_0}{\sqrt{\omega_0^2 - \gamma^2}}
	\frac{\sin\left[ \sqrt{\omega_0^2 - \gamma^2} t - \delta \right]}{\sqrt{\omega_0^2 +\alpha^2 - 2\alpha \gamma}}
	e^{-\gamma t}
	+
	\frac{F_0}{\omega_0^2 + \alpha^2 - 2\alpha\gamma} e^{-\alpha t} \ .
	\label{eq:sol:f:exp}
\end{align}
Here we've defined
\begin{align}
	\tan \delta = \frac{\sqrt{\omega_0^2 - \gamma^2}}{\alpha - \gamma} \ .
\end{align}
If you don't believe this solution, check it by plugging into (\ref{eq:ODE}). Write out the limiting form when the damping goes to zero, $\gamma \to 0$. Show that in the limit of negligible damping and for `late times', that
\begin{align}
	x(t) = \frac{F_0}{\omega_0}\frac{\sin (\omega_0 t -\delta)}{\sqrt{\omega_0^2 + \alpha^2}} \ .
\end{align}
What does `late times' mean in this context? Identify the quantity with dimension $T$ (time) that you can use to define `late.' What does this mean physically? Explain what's happening as one of the terms in (\ref{eq:sol:f:exp}) vanishes.

\subsection{Energy of the system}

If the system initially has zero energy---confirm that this is consistent with the boundary conditions that set $A=B=0$---show that the energy of the system at late times is
\begin{align}
	E = \frac{F_0^2}{2(\omega_0^2 + \alpha^2)}\ .
\end{align}
\textsc{Hint}: recall that the energy for the system is $E = \frac 12 \dot x^2 + \frac 12 \omega_0^2 x^2$.




\section{The Wave Equation in (3+1) Dimensions}
\label{problem:wave:eq:3d}

In this problem, we will review the steps in Lecture 19 and apply them to an actual electromagnetic source. 
%
We solve for the Green's function of wave equation for electromagnetism: 
\begin{align}
	\left[\frac{1}{c^2}\frac{\partial^2}{\partial t^2} - \frac{\partial^2}{\partial \vec{r}^2}\right] 
	\varphi(\vec{r},t) &= \rho(\vec{r},t)
	&
	\left[\frac{1}{c^2}\frac{\partial^2}{\partial t^2} - \frac{\partial^2}{\partial \vec{r}^2}\right]
	\vec A(\vec{r},t) &= \vec j(\vec{r},t) \ .
	\label{eq:EM:wave}
\end{align}
We make the speed of light, $c$, explicit and have chosen \textbf{Lorentz gauge} and are working in vacuum, $\varepsilon_0 = \mu_0 = 1$. 
%
You should recognize all the physical quantities here. 
% 
Use spacetime coordinates such that $x = (ct,\vec r)$. Define the 4-vector potential $A_\mu(x) = \left(\varphi(x),\vec A(x)\right)$, where $x$ is a point in Minkowski space. Let $G(x,x')$ be the Green's function for each component of $A_\mu$. 
%
This means that the potential is given by
\begin{align}
	A_\mu(x) = \int d^4x \, G(x,x') \, j_\mu(x') \ .
\end{align}
Make sure you're comfortable with the indices\footnote{Be able to answer the following questions for yourself: Why doesn't $G(x,x')$ have indices? Why shouldn't it be a 4-component object or a matrix?}. 

In comparison to what we did in lecture, we'll keep factors of $c$ explicit and will write out the space and time components separately rather than writing $\partial^2 = \partial_t^2 -\partial_x^2 -\partial_y^2 -\partial_z^2$.




\subsection{Fourier Transform}

The Green's function equation for each component is
\begin{align}
	\left[\frac{1}{c^2}\frac{\partial^2}{\partial t^2} - \frac{\partial^2}{\partial \vec{r}^2}\right] G(x,x') = \delta^{(4)}(x-x') \equiv \delta(x-x')\delta(y-y')\delta(z-z')\delta(t-t') \ .
	\label{eq:4D:define:G}
\end{align}
We write the Fourier transform in Minkowski space \emph{covariantly}:
\begin{align}
	\tilde G(k,x') &= \int d^4x \, e^{ik\cdot x} G(x,x')
	&
	G(x,x') &= \int \dbar^4k \, e^{-ik\cdot x} \tilde G(k,x') \ ,
\end{align}
where we use the notation $\dbar = d/2\pi$. Further, $k = (E/c,\vec k)$ and we recall that $k\cdot x \equiv k_\mu x^\mu = Et - k_x x - k_y y - k_z z$. Why is $k_0$ identified $E$? If you're not sure, wait until later and see what $E$ is fixed to be.

Solve for $\tilde G(k,x')$. It should look very similar, perhaps with factors of $c$ as required by dimensional analysis.  Write $G(x,x')$ as a 4D integral over $\tilde G(k,x')$.

\textsc{Hint:} It may be useful to recall that $\delta^{(4)}(x)$ has the following Fourier transform: 
\begin{align}
	\delta^{(4)}(x-x') &= \int \dbar^4 k e^{-ik\cdot (x-x')} \ .
\end{align}

\textsc{Answer:} You will find
\begin{align}
	G(x,x') = \int \dbar^4k \, \frac{c^2}{c^2 \vec k^2 - E^2} e^{-ik\cdot (x-x')} \ .
	\label{eq:G:ft}
\end{align} 
Observe that $G(x,x') = G(x-x')$, as we expect from spacetime translation invariance.



\subsection{Angular integrals in hyper-cylindrical coordinates}

For convenience, write $y \equiv x-x' = (cu,\mathbf{s})$. This means that $u$ is a shifted time coordinate and $\vec s$ is a shifted spatial 3-vector. Note that $d^4x = d^4 y$. In order to integrate (\ref{eq:G:ft}), use 4D `hyper-cylindrical coordinates' over $k$ where time/energy is treated linearly and the space/momentum directions are treated in 3D spherical coordinates, $|\vec k|$, $\cos \theta$, and $\varphi$. Recall that $\theta$ is the azimuthal angle with respect to the $k_z$-axis. The volume element in these coordinates is
\begin{align}
	d^4 k = dE \, d^3\vec k = |\vec k|^2 \,dE \,  d|\vec k| \, d\cos \theta \, d\varphi \ .
\end{align}

Since we integrate over all values of $k^\mu$, we are free to align our axes however we want. A particularly convenient choice is to align $k_z$ to be aligned with $\vec s$. In this case, $\vec k \cdot \vec s = ks \cos \theta$, where it should be understood that $k$ and $s$ are magnitudes of the spatial $3$-vectors\footnote{That is, from now on I write $k = |\vec k|$. There should not be any ambiguity with the 4-vector, $k^2 = E^2/c^2 - \vec k^2$, which no longer shows up in our expressions.}.
%
Then we have
\begin{align}
	-ik\cdot y = i k s \cos \theta  - i E u \ .
\end{align}
Perform the angular integrals in (\ref{eq:G:ft}).  Show that this gives
\begin{align}
	G(y) = \frac{c^2}{4\pi^3 s} \int_0^\infty \sin\, ks 
	\left(
	 \int_{-\infty}^\infty \frac{k}{c^2k^2 - E^2} e^{-iE u} dE
	\right) \, dk \ .
	\label{eq:G:cauchy:P:problem}
\end{align}






\subsection{Practical Pole Pushing for Poor Physics People}

The $dE$ integral in parenthesis in (\ref{eq:G:cauchy:P:problem}) looks very familiar. We evaluate it using a contour integral. Identify the location of the two poles in the complex $E$ plane.
%
You're already an expert on this integral.
%
You have a choice of contours: $\mathcal C_+$ in the upper half plane and $\bar{\mathcal C}_-$ in the lower half plane. These are parameterized by:
\begin{align}
	\mathcal C_+ &: \quad z = Re^{i\theta} \qquad  0 < \theta < \pi \\
	\bar{\mathcal C}_- &: \quad z = Re^{i\theta} \qquad \pi < \theta < 2 \pi \ ,
\end{align}
where in each case we take $R\to \infty$.
Comment on which contour one should choose depending on the sign of $u\equiv t-t'$.

Now we have to pick a prescription for how to navigate the poles. There are two physically-motivated choices: one can push the poles into the upper half plane and into the area enclosed by $\mathcal C_+$, or one can push the poles into the lower half plane and into the area enclosed by $\bar{\mathcal C}_-$. Which one corresponds to the causal case? Perform the $dE$ integral for this case.


\subsection{The Green's Function}

You now have $G(y)$ as a single integral over $dk$. Go ahead and perform this integral. 

\textsc{Hints}: Use, once again, the fact that $\int e^{ikx} dk = 2\pi \delta(x)$. The most straightforward calculation involves writing the solution to the $dE$ integral as a sum of exponentials rather than a sine. 

\textsc{Intermediate step}: It is useful to show that
\begin{align}
	\int_{-\infty}^\infty \frac{e^{-iEu}}{(E-ck - i\varepsilon)(E+ck - i\varepsilon)} dE = 
	\frac{-i\pi}{ck} \left(e^{icku}-e^{-icku}\right) \ .
\end{align}
Another useful tip is to use
\begin{align}
	\int_0^\infty \left(e^{ikX} + e^{-ikX}\right) dk = 
	\int_{-\infty}^\infty e^{ikX} dk=
	2\pi \delta(X)
	 \ .
\end{align}


\textsc{Intermediate Step}: Writing $s = |\vec s|$, you should find:
\begin{align}
	G^{(r)}(y) &=
	\frac{c}{4\pi r}
	\left\{
	\begin{array}{ll}
%		\frac{c}{4\pi r} 
%		\left[
		\delta(s-cu) - \delta(s+cu)
%		\right]
		&
		\quad\text{if $u=t-t' \geq 0$}
		\\
		0 & \quad\text{if $u=t-t' <0$} \ .
	\end{array}
	\right.
	\label{eq:Gr:Gs}
\end{align}
Then observe that the radial coordinate $s\geq 0$ so that one of the $\delta$ functions in each expression will always be zero. 

\textsc{Answer}:
Taking $x' = 0$ so that $u=t$ and $s = r$, the final expression is:
\begin{align}
	G^{(r)}(x) = \frac{c}{4\pi r} \delta(r-ct) \Theta(t) \ .
	\label{eq:3d:G}
\end{align}
The physical interpretation is that the Green's function solves the electromagnetic wave equation for a `blip' impulse at $t=0$ and $\vec{r} = 0$. 




 





\section{The Wave Equation in Flatland}

There's a fantastic novella written in 1884 called \emph{Flatland: A Romance of Many Dimensions}\footnote{\url{https://en.wikipedia.org/wiki/Flatland}; see also \url{http://www.geom.uiuc.edu/~banchoff/Flatland/}}. It is a story about a 2+1 dimensional universe that doubled as a social commentary. In this problem, we explore electromagnetic waves in such a universe. 

\textsc{Note}: We walked through most of this problem in Lecture 20, feel free to follow along with the lecture notes.

%\subsection{Green's Function: set up}

We follow the same procedure as the previous problem.
\begin{align}
	\left[\frac{1}{c^2}\frac{\partial^2}{\partial t^2} - \frac{\partial^2}{\partial \vec{r}^2}\right] 
	\varphi(\vec{r},t) &= \rho(\vec{r},t)
	&
	\left[\frac{1}{c^2}\frac{\partial^2}{\partial t^2} - \frac{\partial^2}{\partial \vec{r}^2}\right]
	\vec A(\vec{r},t) &= \vec j(\vec{r},t) \ .
	\label{eq:EM:wave}
\end{align}
The Green's function for each component is:
\begin{align}
	\left[\frac{1}{c^2}\frac{\partial^2}{\partial t^2} - \frac{\partial^2}{\partial \vec{r}^2}\right] G(x,x') = \delta^{(3)}(x-x') \equiv \delta(x-x')\delta(y-y')\delta(t-t') \ .
	\label{eq:HO:in:3d}
\end{align}

\subsection{Fourier Transform}

Solve for the Green's function using a Fourier transform,
\begin{align}
	\tilde G(k,x') &= \int d^3x \, e^{ik\cdot x} G(x,x')
	&
	G(x,x') &= \int \dbar^3k \, e^{-ik\cdot x} \tilde G(k,x') \ ,
\end{align}
where continue to use the notation $\dbar = d/2\pi$. Further, $k = (E/c,\vec k)$ and we recall that $k\cdot x \equiv k_\mu x^\mu = Et - k_x x - k_y y$. Also note the $\delta$ function in 3-space: $\delta^{(3)}(x-x') = \int \dbar^3 k e^{-ik\cdot (x-x')}$ \ .

\textsc{Answer:} It should be no surprise to you at all that you end up with the following:
\begin{align}
	G(x,x') = \int \dbar^3k \, \frac{c^2}{c^2 \vec k^2 - E^2} e^{-ik\cdot (x-x')} \ .
	\label{eq:G:ft}
\end{align} 


\subsection{Something familiar}

For convenience, write $y \equiv x-x' = (cu,\mathbf{s})$. Check that it is still true that
\begin{align}
	-ik\cdot y = i k s \cos \theta  - i E u \ .
\end{align}
Unlike Problem~\ref{problem:wave:eq:3d}, however, our integration measure is
\begin{align}
	d^3 k = dE \, d^2\vec k = |\vec k| \,dE \,  d|\vec k| \, d \theta  \ .
\end{align}
\textsc{Warning:} At this stage, you have plenty of options for which integral to do first. Mathematically, it doesn't matter which order you do them---but some paths have more tedious integrals than others. I'm going to give hints for doing the integrals in the following order: $dE$, $d|\vec k|$, $d\theta$. 


Go ahead and perform the $dE$ integral for the case of the retarded Green's function. \textsc{Hint}: hey, haven't you done this \emph{exact} integral before? 

\textsc{Answer}:
\begin{align}
\int dE \frac{e^{-iEt}}{c|\vec k|^2 -E^2}
&=
\frac{-i\pi}{c|\vec k|} 
\left(
e^{ic|\vec k| t}
-
e^{-ic|\vec k| t}
\right) \ .
\label{eq:dE}
\end{align}

\textsc{Comment}: At this point, you might want to make a mental note of where $i\varepsilon$'s would be if we were keeping track of them.

\subsection{The $d|\vec k|$ integral}

Now perform the $d\theta$ integral. I suggest doing this the following way: write the result of (\ref{eq:dE}) into a sine function. Then use the relation
\begin{align}
	\int_0^\infty e^{iAk}\sin(ckt)\, dk = \frac{ct}{-A^2 + c^2t^2} \ .
\end{align}
\textsc{Remark:} This is a slightly mathematically dubious step which is okay if we judiciously kept track of the $i\varepsilon$. Specifically, this only holds when $t > r|\cos\theta|$. We'll be sloppy about this for now, we'll see the actual constraints come out more cleanly below.

\textsc{Answer}: You will find:
\begin{align}
	G(y) = \frac{1}{(2\pi)^2} \int d\theta \frac{ct}{c^2t^2 - r^2\cos^2\theta} \ .
\end{align}

\subsection{The angular integral}

A useful trick here is to use
\begin{align}
	\int_0^{2\pi} \frac{d\theta}{1-B\cos^2\theta} = 2\pi \sqrt{\frac{1}{1-B}} \ .
\end{align}
This only holds when $B<1$. 

\textsc{Answer}: You will find
\begin{align}
	G(y) = \frac{1}{2\pi} \frac{c}{\sqrt{c^2 t^2 - s^2}} \ ,
	\label{eq:2d:G:hardway}
\end{align}
with the caveat that $c^2t^2 > s^2$. Convince yourself that this makes sense from the point of view of causality.


\subsection{Dimensional Reduction}

% Now do it the Appel way.

The result (\ref{eq:2d:G:hardway}) solves the problem. 
%
There's actually a much nicer way to get the lower-dimensional Green's function from a higher-dimensional Green's function. 

Prove that 
\begin{align}
	G(x,y,t) = \int_{-\infty}^\infty G_{(3)}(x,y,z,t) \, dz \ ,
	\label{eq:G:from:G3}
\end{align}
where $G_{(3)}$ is the (3+1)-dimensional Green's function, (\ref{eq:3d:G}). In other words: one can derive the (2+1)-dimensional Green's function by integrating over the third dimension.

We prove this result without using the explicit form of $G_{(3)}(x,y,z,t)$, just the defining relation (\ref{eq:4D:define:G}). Write the right-hand side of (\ref{eq:G:from:G3}) as a Fourier transform, then do the $dz$ integral:
\begin{align}
	\int_{-\infty}^\infty dz G_{(3)}(x,y,z,t)dz
	&=
	\int \dbar^4k \,
	\left[dz\; e^{ikz} \tilde G_{(3)}(E,k_x, k_y, k_z)\right]
	 e^{-iEt} e^{i(k_xx + k_y y)} \ .
\end{align}
The $dz$ integral is just an inverse Fourier transform that gives $2\pi\delta(k_z)$. Perform the $k_z$ integral to write the right-hand side of (\ref{eq:G:from:G3}) in the form of a (2+1)-dimensional Fourier transform: $\int \dbar^3 k \, \tilde{\mathfrak{g}}e^{-iEt}e^{i(k_xx+k_yy)}$. 

Writing $\tilde{\mathfrak{g}}$ as the Fourier transform of the right-hand side of (\ref{eq:G:from:G3}), show that (\ref{eq:4D:define:G}) implies
\begin{align}
	\left(E^2 - k_x^2 - k_y^2\right) \tilde{\mathfrak{g}} = 1\ ,
\end{align}
thus proving that $\tilde{\mathfrak{g}}$ is the Fourier transform of the (2+1)-dimensional Green's function.

\subsection{Dimensional Reduction: explicit result}

For the retarded Green's function in (3+1)-dimensions, we found
\begin{align}
	G_{(3)}(\vec r,t) &= \frac{c}{4\pi r} \delta(r-ct)
	&
	\text{for } t&>0 \ ,
\end{align}
and zero for $t<0$. Integrate this expression over $dz$ to obtain the (2+1)-dimensional Green's function, $G(x,y,t)$,
\begin{align}
	G(x,y,t) &= \frac{c}{4\pi} \int_{-\infty}^\infty 
	\frac{1}{\sqrt{x^2+y^2 + z^2}} \delta\left(\sqrt{x^2+y^2 + z^2} - ct\right) \, dz \ .
\end{align}

\textsc{Hint:} you have one integral to do, but fortunately you have a $\delta$-function to do it. The only problem is that the $\delta$ function is of the form $\delta\left(f(z)\right)$. One way to solve this is to do a change of variables to $w = \sqrt{x^2+y^2+z^2}$ so that the $\delta$-function is of the form $\delta(w-ct)$. Doing this more generally gives a convenient rule,
\begin{align}
	\delta\left(f(z)\right) &= \sum_{z_i} \frac{1}{|f'(z_i)|} \delta(z-z_i)
	&
	z_i \text{ such that } f(z_i) = 0 \ .
\end{align}

\textsc{Answer:} The retarded Green's function is:
\begin{align}
	G(x,y,t)  &= 
	\left\{
	\begin{array}{ll}
		\frac{c}{2\pi} \frac{1}{\sqrt{c^2 t^2 - x^2 - y^2}}
		& \text{if } c^2t^2 > x^2 + y^2\\
		0 & \text{otherwise}
	\end{array}
	\right. \ .
\end{align}


\subsection{Discussion: The Odd Electrodynamics of Flatland}

Sketch $G(r,t)$ as a function of $r = \sqrt{x^2 + y^2}$ for two values of $t$. Compare this to the corresponding plot for $G_{(3)}(r,t)$. 

Think about how strange this lower-dimensional universe is! In our universe, if you used a flash camera in the dark, the each part of the room would light up for a brief instant corresponding to when the photons from the flash reaches each part of the room. In this lower-dimensional universe, $G(r,t)$ is telling us that the entire room remains illuminated for an \emph{infinite} amount of time with a rapidly decreasing (but never strictly zero) intensity.











\appendix
\section*{\Large Extra Credit}


These problems are not graded and are for your edification. You are strongly encouraged to explore and discuss these topics, especially if they are in a field of interest to you.

\section{Invariance of the Green's Function}

\subsection{Invariance of that funny $\delta$-function}

In Problem~\ref{problem:wave:eq:3d}, you might be concerned about is the covariance of the $\delta$-function with respect to Lorentz transformations. The Green's function $G(x)$ is supposed to be invariant under Lorentz transformations. However, the argument of $\delta(ct - r)$ does not appear to Lorentz invariant. As a reminder, under a Lorentz transformation $x^\mu \to \Lambda^\mu_{\phantom\mu\nu}x^\nu$ (this is matrix multiplication). However, the scalar quantity $x^2 = x_\mu x^\mu$ remains unchanged (invariant).

Show that the Green's function is indeed invariant.


\textsc{Hint:} Recall that the following rule for $\delta$-functions:
\begin{align}
	\delta\left(g(x)\right) &= \sum_{x_i} \frac{\delta(x-x_i)}{|g'(x_i)|}
	&
	g(x_i) = 0 \ .
\end{align}
Writing $x = (ct,\vec r)$, show that
\begin{align}
	\delta(x^2) = \frac{1}{2r} \left[
	\delta(r-ct) - \delta(r+ct)
	\right] \ .
\end{align}
Compare this to (\ref{eq:Gr:Gs}). 

\textsc{Answer}: You will find that 
\begin{align}
	G^{(r)}(x) = \frac{c}{2\pi} \delta(x^2) \Theta(t)
\end{align}
where $\Theta(t)$ is the Heaviside step function. Argue that even though $t=x^0$ is itself not invariant, $\Theta(t)$ is invariant under any \emph{physical} Lorentz transformation. 


\subsection{Discussion}
% Appel ex. 8.9
It turns out that when calculating matrix elements (amplitudes) for quantum mechanical scattering a relativistic particle of mass $m$ with four-momentum $p = (E,\vec p)$, you often end up with expressions in momentum space where an integrand $f(p)$ needs to be integrated over phase space in a way where the on-shell conditions are fixed:
\begin{align}
	p^2 &= E^2 - \vec p^2 = m^2 
	&
	E &>0
	\ .
\end{align}
In other words: you have an integral over the four directions in momentum space, but `physicality' imposes a constraint. You should be dreaming of Lagrange multipliers\footnote{We won't use Lagrange multipliers, but a generalization of this method where the $\delta$-function is promoted to a Lagrange multiplier shows up in gauge theory and is called the Fadeev-Popov procedure.}. 

What one ends up doing is writing down integrals of the form
\begin{align}
	\int d^4p \, \delta(p^2 - m^2) \Theta(E) f(p) \ .
\end{align}
This says: \emph{integrate over all of the phase space for $p^\mu$, but restrict it to positive energies and to 4-momenta that satisfy the $p^2=m^2$.}

Writing $E(\vec p) = \sqrt{\vec p^2 +m^2}$, show that
\begin{align}
	\int d^4p \, \delta(p^2 - m^2) \Theta(E) f(p) \ .
	& = 
	\int \frac{d^3\vec p}{2E(\vec p)} f(E(\vec p), \vec p) \ .
\end{align}
This funny-looking differential element $d^3\vec p/2E(\vec p)$ doesn't look Lorentz invariant, but we have derived that it is. Note that in the function, $f$, we fix $E$ to be the required value by `on-shell-ness.'




\section{Dimensional Regularization}

% e.g. Wikipedia, Cahill
% contour integral derivation
% do volumes of spheres in N dimensions
% Cahill p. 188

The Euler $\Gamma$ function is a meromorphic generalization of the factorial,
\begin{align}
	\Gamma(z) = \int_0^\infty x^{z-1}e^{-x} dx \ .
	\label{eq:Gamma}
\end{align}
For positive integers, $\Gamma(n) = (n-1)!$.
A fact that shows up occasionally in physics that area of the unit sphere in $d$ dimensions is
\begin{align}
	S_d = \frac{2\pi^{d/2}}{\Gamma(d/2)} \ .
	\label{eq:Sd}
\end{align}
There's always ambiguity (to me at least) when people say things like ``$d$-sphere'' ... is this a sphere embedded in $d$ dimensions whose area has units $L^{d-1}$, or is this the sphere embedded in $(d+1)$ dimension whose area has units $L^d$? What is the correct interpretation of (\ref{eq:Sd})? What is the area of the $d$-sphere of radius $r$? What is the [hyper]-volume enclosed by this sphere? (This is discussed in Cahill's textbook, around exercise 4.12.)

% Cahill ex 4.12


In physics, one often has to deal with quantifying and understanding apparent divergences in our calculations\footnote{These non-analytic results are the sign of our theory trying to tell us something.}. A systematic treatment of how to do this is usually buried deep in a course of quantum field theory, where there is a danger of confusing the physics from the mathematical techniques. In this problem we explore one such technique called \textbf{dimensional regularization} in a very simple electrostatic system. 

Consider the case of an infinite line  with constant charge density $\lambda$. (This should remind you of the dimensional reduction in Problem 3!) Pick coordinates such that the charged line is along the $y$-axis, and the observer is on the $x$-axis\footnote{By rotational symmetry $x$ may as well be the radial cylindrical coordinate, $\rho$.}. Observe that the potential (an integral over the Green's function!) at $x$ is
\begin{align}
	V(x) = \frac{\lambda}{4\pi} \int_{-\infty}^\infty \frac{dy}{\sqrt{x^2+y^2}} \ .
	\label{eq:Vx:1d}
\end{align}
Note that this integral is divergent\footnote{The expression is also scale invariant: $V(x) = V(kx)$. Think about what this means physically: an infinite line charge looks the same when you're close to it versus when you are far from it. This makes sense: you have no other length scale to compare to---thus by dimensional analysis, you have no basis to even use the words `close' or `far.' Think about implications Problem 3.}. What does this mean and how do we make sense of quantities like $\vec E = -\nabla V$ in this case? As physicists, we deal with infinite quantities by \textbf{regulating} them: we \emph{quantify} the infinity by some parameter which allows us to compute finite physical quantities like $\vec E$ and check that those physical quantities are independent of the regulator.

Regulate the $dy$ integral (\ref{eq:Vx:1d}) by converting the one-dimensional integral into a ($1-2\epsilon$)-dimensional integral, where $\epsilon$ is a small number. Don't stop to ask what this means\footnote{\url{https://www.youtube.com/watch?v=ItV8utelYlc}. You can (and should) think about what this `slightly less than one dimension' means after you see the end result.}. This means we write:
\begin{align}
	V(x) = \frac{\lambda}{4\pi} \int_0^\infty dy\,d\Omega_{(1-2\epsilon)} \frac{y^{-2\epsilon}}{\mu^{-2\epsilon}}\frac{1}{\sqrt{x^2+y^2}} \ ,
\end{align}
where $d\Omega_{(1-2\epsilon)}$ is the differential area of the unit sphere in $(1-2\epsilon)$ dimensions, the factor of $y^{-2\epsilon}$ is inserted the give the right measure for a $(1-2\epsilon)$-dimensional volume, and some \emph{arbitrary} dimensionful parameter $\mu$ has been introduced to make units work out\footnote{I am using units such that $[V] = [\lambda]$. You can put in $\epsilon_0$ as appropriate, if you really wish.}. 
\begin{enumerate}[(a)]
\item Show that this integral evaluates to
\begin{align}
	V(x) = \frac{\lambda}{4\pi} \frac{\mu^{2\epsilon}}{x^{2\epsilon}} \frac{\Gamma(\epsilon)}{\pi^\epsilon} \ .
\end{align}
\item Confirm that as $\epsilon\to 0$, $V(x)$ becomes ill defined. \textsc{Hint}: Recall from $\Gamma(z+1) = z \Gamma(z)$ that $\Gamma(z)$ has a simple pole at the origin.
\item Calculate the electric field in the $x$ direction; $E_x = -\partial_xV(x)$ and show that
\begin{align}
	E_x = \frac{\lambda}{4\pi} \left[ \frac{2\epsilon \mu^{2\epsilon} \Gamma(\epsilon)}{\pi^\epsilon x^{1+2\epsilon}} \right] \ .
\end{align}
\item Take the $\epsilon \to 0$ limit and observe that $E_x = \lambda/2\pi x$, which is independent of the arbitrary scale $\mu$ and behaves the way that one expects. 
\end{enumerate}

\textsc{Hint:} Use the relation
\begin{align}
	\int_0^\infty \frac{y^{n-1}dy}{\sqrt{x^2+y^2}} = \frac{1}{2}\frac{\Gamma(\frac n2)\Gamma(\frac{1-n}2)}{\sqrt{\pi}}x^{n-1} \ .
\end{align}
This problem is worked out in \url{http://arxiv.org/abs/0812.3578}; you are strongly encouraged to read and think carefully about the entire article.



\end{document}